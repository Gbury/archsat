% $Id$

\section{Conclusion}

We have described the architecture of \archsat{}, an automated theorem prover
that combines a SAT solver with tableau calculus and rewriting. The tableau
calculus is introduced as a regular theory of SMT solving, and the rules are
used to unfold propositional content into clauses while atomic formulas are
handled using satisfiability decision procedures as in SMT solvers. To deal with
quantified first order formulas, we use metavariables and perform rigid
unification modulo equalities and rewriting, for which we introduce an algorithm
based on superposition, but where all clauses contain a single atomic formula.
As for rewriting, it is introduced along the lines of deduction modulo theory,
where axioms are turned into rewrite rules over both propositions and terms.

As perspectives, we plan to realize more tests of \archsat{} over other theories
where a large part of these theories can be turned into rewrite rules. In
particular, a regular trigger mechanism has been implemented in \archsat{} and
can be used to deal with conditional rewriting (the instantiation is delayed and
performed once the condition has been evaluated to true). This feature should 
open up a range of new perspectives on the theories that our approach could
handle. We also aim to apply our tool to the benchmark of the \bware{}
project~\cite{BWare}, which consists of a large collection of proof obligations
coming from the development of industrial applications using the \bmth{} method.
This collection gathers about 13,000~problems, and should allow us to understand
to what extent our tool scales up. To do so, we have to extend \archsat{} to
arithmetic, which can be done using the modular architecture of \archsat{} and
adding an arithmetic theory to the SAT solver core of the prover.
