% $Id$

\documentclass[a4paper]{easychair}

\usepackage[utf8]{inputenc}
\usepackage[T1]{fontenc}
\usepackage{makeidx}
\usepackage[hidelinks]{hyperref}
\usepackage{bussproofs}
\usepackage{amsmath,amsfonts,amssymb}
\usepackage{stmaryrd}
\usepackage{tabularx}
\usepackage{pgf,tikz}
\usepackage{url}
\usepackage{float}
\usepackage{color}
\usepackage{cite}
\usepackage{booktabs}
\floatstyle{boxed}
\restylefloat{figure}
\usepackage{relsize}

% $Id$

\def\ai{\textsf{Aesthetic Integration}}
\def\altergoz{\textsf{Alt-Ergo~Zero}}
\def\archsat{\textsf{ArchSAT}}
\def\atelierb{\textsf{Atelier~B}}
\def\bbook{\textsf{B}-Book}
\def\bmth{\textsf{B}}
\def\ens{\textsf{ENS Paris-Saclay}}
\def\inria{\textsf{Inria}}
\def\lirmm{\textsf{LIRMM}}
\def\lsv{\textsf{LSV}}
\def\msat{\textsf{mSAT}}
\def\um{\textsf{Université de Montpellier}}

\EnableBpAbbreviations{}
\newcommand{\UICm}[1]{\UIC{$#1$}}
\newcommand{\AXCm}[1]{\AXC{$#1$}}
\newcommand{\BICm}[1]{\BIC{$#1$}}
\newcommand{\TICm}[1]{\TIC{$#1$}}
\newcommand{\RLm}[1]{\RL{$#1$}}

\def\rew{\longrightarrow}
\def\arg{\mbox{-}}
\def\type{\textsf{Type}}
\def\omicron{o}
\newcommand\set[1]{\mathsf{set}(#1)}
\newcommand\tuple[2]{\mathsf{tup}(#1,#2)}
\def\plus{\raisebox{.22\height}{\scalebox{.6}{\pmb{+}}}}
\def\fctpart{~{\kern+.2em\mapstochar{}\!\!\!\!\!\rightarrow}~}
\def\injpart{~~{\mapstochar{}\!\!\!\!\!\!\rightarrowtail}~}
\def\surpart{~{\kern+.2em\mapstochar{}\!\!\!\!\!\twoheadrightarrow}~}
\def\bijpart{~{\kern+.1em\mapstochar{}\!\!\!\!\!\rightarrowtail\kern+.03em
\!\!\!\!\!\!\twoheadrightarrow}~}
\def\bijtotal{~{\kern-.1em\rightarrowtail\kern+.03em\!\!\!\!\!\!
\twoheadrightarrow}~}
\def\override{~~{\plus{}\mkern-24mu<}~}
\def\substleft{~{-\mkern-14mu\lhd}~}
\def\substright{~{-\mkern-14mu\rhd}~}

\newlength\memparindent
\setlength{\memparindent}{\parindent}
\newcommand{\myindent}{\hspace{\memparindent}}

\newcommand\todo[1]{\textcolor{red}{#1}}


\begin{document}

\title{SMT Solving Modulo Tableau and Rewriting Theories}
\titlerunning{SMT Solving Modulo Tableau and Rewriting Theories}

\author{Guillaume~Bury\inst{1} \and Simon~Cruanes\inst{2} \and
David~Delahaye\inst{3}}
\authorrunning{Guillaume~Bury, Simon~Cruanes, and David~Delahaye}

\institute{\inria{}/\lsv{}/\ens{}, Cachan, France\\
\email{Guillaume.Bury@inria.fr} \and
\ai{}, Austin (Texas), USA\\
\email{simon@aestheticintegration.com} \and
\lirmm{}/\um{}, Montpellier, France\\
\email{David.Delahaye@lirmm.fr}}

\maketitle

\begin{abstract}
We propose an automated theorem prover that combines an SMT solver with tableau
calculus and rewriting. Tableau inference rules are used to unfold propositional
content into clauses while atomic formulas are handled using satisfiability
decision procedures as in traditional SMT solvers. To deal with quantified first order
formulas, we use metavariables and perform rigid unification modulo equalities
and rewriting, for which we introduce an algorithm based on superposition, but
where all clauses contain a single atomic formula. Rewriting is introduced along
the lines of deduction modulo theory, where axioms are turned into rewrite rules
over both terms and propositions. Finally, we assess our approach over a
benchmark of problems in the set theory of the \bmth{} method.
\end{abstract}

% $Id$

\section{Introduction}

Over the last past few
years, SMT solvers have appeared as very efficient tools to reason over some
well identified theories (equality, uninterpreted functions, linear arithmetic,
arrays, etc.), and have allowed us to bring SAT solving toward first order logic.
Although modern SMT solvers support first order logic, most of them use
heuristic quantifier instantiation for incorporating quantifier reasoning with
ground decision procedures. This mechanism is relatively effective in some cases
in practice, but it is not refutationally complete for first order logic. Hints
(triggers) are usually required, and it is sensitive to the syntactic structure
of the formula, so that it fails to prove formulas that can be easily discharged
by provers based on more traditional first order proof search methods (tableaux,
resolution, etc.).

In this paper, we propose to improve first order proof search by introducing
rewriting into the SAT solver as a regular SMT theory, along the lines of
deduction modulo theory. Deduction modulo theory~\cite{DA03} focuses on the
computational part of a theory, where axioms are transformed into rewrite rules,
which induces a congruence over propositions, and where reasoning is performed
modulo this congruence. In deduction modulo theory, this congruence is then
induced by a set of rewrite rules over both terms and propositions. In our
proposal, this congruence is used a first time to speed up ground reasoning
by computing normal forms for terms, but this still yields an incomplete
algorithm.

We thus propose to overcome the problem of completeness for first
order logic by using tableau calculus as an SMT theory. The tableau calculus
rules are used to unfold propositional content into clauses while atomic
formulas are handled using satisfiability decision procedures as in
regular SMT solvers. To deal with quantified first order formulas, we use
metavariables and perform rigid unification modulo equalities and modulo rewriting,
for which we introduce an algorithm based on superposition, but where all clauses
contain a single atomic formula.

Our approach provides several advantages compared to usual SMT solving and first
order proof search methods. First, we benefit from the efficiency of a SAT
solver core together with a complete method of instantiation (when a
propositional model is found, we try to find a conflict between two literals by
unification). Second, it should be noted that our approach requires no change in
the architecture of the SMT solver, since the tableau calculus and rewriting are
seen as regular theories. Finally, no preliminary Skolemization and Conjunctive
Normal Form (CNF) transformation is required. This transformation is performed
lazily by applying the tableau rules progressively when a literal is propagated
or decided. This makes the production of genuine output proofs easier, contrary
to the usual approach, where the Skolemization/CNF translation is realized at
the beginning and externalized with respect to the proof search.

Our proposal combining SMT solving with tableau calculus and rewriting has been
implemented and the corresponding tool is called \archsat{}. This tool is able
to deal with first order logic extended to polymorphic types à la ML, through a
type system in the spirit of~\cite{BP13}. To test this tool, we propose a
benchmark in the framework of the set theory of the \bmth{}
method~\cite{B-Book}. This theory~\cite{BA15} has been expressed using first
order logic extended to polymorphic types and turned into a theory that is
compatible with deduction modulo theory, i.e. where a large part of axioms has
been turned into rewrite rules. The benchmark itself gathers 319~lemmas coming
from Chap.~2 of the \bbook{}~\cite{B-Book}.

The paper is organized as follows: in Sec.~\ref{sec:smt}, we introduce the tableau
and rewriting theories; we then describe, in Sec.~\ref{sec:super}, our mechanism
of equational reasoning by means of rigid unit superposition; finally, in
Sec.~\ref{sec:bench}, we present some experimental results obtained by running
our implementation over a benchmark of problems in the \bmth{} set theory.

% $Id$

\section{SAT Solving Modulo Tableau and Rewriting Theories}
\label{sec:smt}

Compared to genuine tableau automated theorem provers, like \princess{} or
\zenon{} for example, our approach has the benefit of being versatile since the
tableau rules are actually integrated as a regular SMT theory. This way, the
tableau rules can be easily combined with other theories, such as equality logic
with uninterpreted functions or arithmetic. The way we integrate the tableau
rules into the SMT solver (mainly by boxing/unboxing first order formulas) is
close to what is done in the \satallax{} tool~\cite{CEB12}. The difference
resides in the fact that we are in a pure first order framework, which has
significant consequences in the management of quantifiers and unification in
particular (see Sec.~\ref{sec:super}).

Regarding the integration of rewriting, automated theorem provers rely on
several solutions (superposition rule for first order provers, triggers for SMT
solvers, etc.). But deduction modulo theory~\cite{DA03} is probably the most
general approach, where a theory can be partly turned into a set of rewrite
rules over both terms and formulas. Several proof search methods have been
extended to deduction modulo theory, resulting in tools such as \iproverm{} and
\zenm{}. This paper can be seen as a continuation of these previous experiments
adapted to the framework of SMT solving.

\subsection{The Tableau Theory}
\label{sec:tab}

We introduce $\mathcal{T}$ and $\mathcal{F}$ respectively, the sets of first
order terms and formulas over the signature
$\mathcal{S}=(\mathcal{S}_\mathcal{F},\mathcal{S}_\mathcal{P})$, where
$\mathcal{S}_\mathcal{F}$ is the set of function symbols, and
$\mathcal{S}_\mathcal{P}$, the set of predicate symbols, such that
$\mathcal{S}_\mathcal{F}\cap\mathcal{S}_\mathcal{P}=\emptyset$. The set
$\mathcal{T}$ is extended with two kinds of terms specific to tableau proof
search. First are $\epsilon{}$-terms (used instead of Skolemization) of the form
$\epsilon(x).P(x)$, where $P(x)$ is a formula, and which means some $x$ that
satisfies $P(x)$, if it exists. And second, metavariables (often named free
variables in the tableau-related literature) of the form $X_P$, where $P$ is the
formula that introduces the metavariable, and which is either $\forall{}x.Q(x)$
or $\neg\exists{}x.Q(x)$, with $Q(x)$ a formula. Metavariables are written using
capitalized letters (such as $X_P$ or $Y_P$), will never be
substituted\footnote{More generally, terms are immutable and are never modified
in place: application of a substitution creates new terms that may be used,
but will never modify existing terms.}. They will also be considered rigid in
the following, meaning that in a context where we try and unify terms, by building
a mapping from metavariables to terms, each metavariables may be bound at most
once.  This is because in tableaux calculus, if you instantiate a formula
multiple times, it may create additional propositional branches, which we do not
want. Instead we consider metavariables rigid, and we'll generate multiple
(distinct) metavariables for the same formula if needed.

A boxed formula is of the form $\lfloor{}P\rfloor$, where $P$ is a formula. A
boxed formula is called an atom, and a literal is either an atom, or the
negation of an atom. A literal is such that there is no negation on top of the
boxed formula (which means that $\lfloor\neg{}P\rfloor$ is automatically
translated into $\neg\lfloor{}P\rfloor$, and $\neg\neg\lfloor{}P\rfloor$ into
$\lfloor{}P\rfloor$).  A clause is a disjunction of literals. It should be noted
that SMT solving usually reasons over sets of clauses composed of first order
terms; here, a literal is a first order formula (possibly with quantifiers),
which requires to box formulas to get a regular SAT solving problem where boxed
formulas are propositional variables.  The state of a SMT solver can be
represented as $T \parallel S$, with $T$ the trail of the SMT, i.e.~a list of
boxed formula, in chronological order left to right, and $S$ the set of clauses
to be satisfied by the solver. One very important point is that SMT theories can
and will often add some clauses to the set $S$ of clauses that the solver tries
to satisfy. This is sound as long as only tautologies are added. In the
following, this will often be abbreviated by only writing ``adding the clause''.

Tableau proof search method is integrated as a regular theory in our SMT solver.
Whenever a literal $l$ is decided or propagated by the solver, the tableaux
theory generates the set of clauses $\llbracket{}l\rrbracket$, where the
function $\llbracket\cdot\rrbracket$ is described by the rules of
Fig.~\ref{fig:tabth}, and then add these clauses\footnote{We use all rules
except the instantiation $\gamma$-rules: $\gamma_{\forall\mathrm{inst}}$ and
$\gamma_{\neg\exists\mathrm{inst}}$ which are only used once adequate terms
for instantiation have been found}.  It should be noted that we use the same
names for the rules as in tableau calculus ($\alpha$-rules, $\beta$-rules,
etc.), but there is no precedence between rules and therefore no priority in the
application of the rules contrary to the tableau proof search method (where
$\alpha$ rules are applied before $\beta$-rules, and so on).  Application of
this can be seen in Figure~\ref{fig:rew}, on the third line, where once the
solver has propagated that $A \equiv \lfloor B \rightarrow C \rfloor$ is false,
the tableaux theory adds two new clauses: $A \lor B$ and $A \lor \neg C$, whose
aim is to ensure that as long as $A$ is false, $B$ will be true and $C$ will be
false.  The clauses added should be seen as an implication: $\neg A \rightarrow
B$ and $\neg A \rightarrow \neg C$, where the presence of $A$ in the clauses is
important because the tableaux theory does not know whether $A$ is always false
or not: it might be that $A$ being false was a decision of the SAT solver (or a
consequence of a decision), and thus might be backtracked; in this case it is
important to not have the propagations of $B$ and $\neg C$ depend on the
propagation of $\neg A$.

When the SMT solver finds a model $M$ of the current set of clauses, we look for
a conflict in $M$ between boxed atomic formulas by unification.  If there exist
two literals $l$ and $\neg{}l'$ in $M$ such that $l=\lfloor{}Q\rfloor$ and
$l'=\lfloor{}R\rfloor$, with $Q$ and $R$ two formulas, then for each binding
$(X_{\forall{}x.P(x)}\mapsto{}t)\in\mathrm{mgu}(Q,R)$ (resp.
$(X_{\neg\exists{}x.P(x)}\mapsto{}t)\in\mathrm{mgu}(Q,R)$) we can generate the
clauses $\llbracket\lfloor\forall{}x.P(x)\rfloor{}\rrbracket$
(resp. $\llbracket\neg\lfloor\exists{}x.P(x)\rfloor{}\rrbracket$) using the rule
$\gamma_{\forall\mathrm{inst}}$ (resp. $\gamma_{\neg\exists\mathrm{inst}}$) of
Fig.~\ref{fig:tabth}, and then add these new clauses. For instance, suppose we
find a model where $\lfloor P(X_{\forall x. P(x)}) \rfloor$ is true and $\lfloor
P(a) \rfloor$ is false, then we'll add the following clause to the solver: $\neg
\lfloor \forall x. P(x) \rfloor \lor \lfloor P(a) \rfloor$, representing the
instantiation of $\forall x. P(x)$ using the term $a$. This clause will ensure
that the model where $\lfloor P(a) \rfloor$ is false cannot happen again as long
as $\lfloor \forall x. P(x) \rfloor$ is true. Note that we did not substitute
any metavariables, we only added some tautologies, thus there is no need for any
special backtracking.

\begin{figure}[t]
\parbox{\textwidth}
{\small
\underline{Analytic Rules}
\begin{center}
$\begin{array}{llll}
(\alpha_\land) &
\llbracket\lfloor{}P\land{}Q\rfloor{}\rrbracket =
\left\{\begin{array}{l}
  \neg\lfloor{}P\land{}Q\rfloor\lor\lfloor{}P\rfloor \\
  \neg\lfloor{}P\land{}Q\rfloor\lor\lfloor{}Q\rfloor
\end{array}\right. &

(\beta_{\neg\land}) &
\llbracket\neg\lfloor{}P\land{}Q\rfloor{}\rrbracket =
\lfloor{}P\land{}Q\rfloor\lor\neg\lfloor{}P\rfloor\lor\neg\lfloor{}Q\rfloor

\\\\

(\beta_\lor) &
\llbracket{}\lfloor{}P\lor{}Q\rfloor{}\rrbracket =
\neg\lfloor{}P\lor{}Q\rfloor\lor\lfloor{}P\rfloor\lor\lfloor{}Q\rfloor &

(\alpha_{\neg\lor}) &
\llbracket\neg\lfloor{}P\lor{}Q\rfloor{}\rrbracket =
\left\{\begin{array}{l}
  \lfloor{}P\lor{}Q\rfloor\lor\neg\lfloor{}P\rfloor \\
  \lfloor{}P\lor{}Q\rfloor\lor\neg\lfloor{}Q\rfloor
\end{array}\right.

\\\\

(\beta_\Rightarrow) &
\llbracket{}\lfloor{}P\Rightarrow{}Q\rfloor{}\rrbracket =
\neg\lfloor{}P\Rightarrow{}Q\rfloor\lor\neg\lfloor{}P\rfloor\lor
\lfloor{}Q\rfloor &

(\alpha_{\neg\Rightarrow}) &
\llbracket\neg\lfloor{}P\Rightarrow{}Q\rfloor{}\rrbracket =
\left\{\begin{array}{l}
  \lfloor{}P\Rightarrow{}Q\rfloor\lor\lfloor{}P\rfloor \\
  \lfloor{}P\Rightarrow{}Q\rfloor\lor\neg\lfloor{}Q\rfloor
\end{array}\right.

\\\\

(\beta_\Rightarrow) &
\llbracket{}\lfloor{}P\Leftrightarrow{}Q\rfloor{}\rrbracket =
\left\{\begin{array}{l}
  \neg\lfloor{}P\Leftrightarrow{}Q\rfloor\lor\lfloor{}P\Rightarrow{}Q\rfloor \\
  \neg\lfloor{}P\Leftrightarrow{}Q\rfloor\lor\lfloor{}Q\Rightarrow{}P\rfloor
\end{array}\right. &

(\beta_{\neg\Rightarrow}) &
\llbracket{}\neg\lfloor{}P\Leftrightarrow{}Q\rfloor{}\rrbracket =
\begin{array}{l}
  \lfloor{}P\Leftrightarrow{}Q\rfloor \\
  \lor\neg\lfloor{}P\Rightarrow{}Q\rfloor \\
  \lor\neg\lfloor{}Q\Rightarrow{}P\rfloor
\end{array}
\end{array}$
\end{center}

\underline{$\delta$-Rules}
\begin{center}
$\begin{array}{lll@{\hspace{0.5cm}}l}
\llbracket\lfloor\exists{}x.P(x)\rfloor{}\rrbracket & = &
\neg{}\lfloor\exists{}x.P(x)\rfloor\lor\lfloor{}P(\epsilon(x).P(x))\rfloor &
(\delta_\exists)

\\

\llbracket\neg\lfloor\forall{}x.P(x)\rfloor{}\rrbracket & = &
\lfloor\forall{}x.P(x)\rfloor\lor\neg\lfloor{}P(\epsilon(x).\neg{}P(x))\rfloor &
(\delta_{\neg\forall})
\end{array}$
\end{center}

\underline{$\gamma$-Rules}
\begin{center}
$\begin{array}{lll@{\hspace{0.5cm}}l}
\llbracket\lfloor\forall{}x.P(x)\rfloor{}\rrbracket & = &
\neg\lfloor\forall{}x.P(x)\rfloor\lor\lfloor{}P(X_{\forall{}x.P(x)})\rfloor &
(\gamma_{\forall{}M})

\\

\llbracket\neg\lfloor\exists{}x.P(x)\rfloor{}\rrbracket & = &
\lfloor\exists{}x.P(x)\rfloor\lor
\neg\lfloor{}P(X_{\neg\exists{}x.P(x)})\rfloor &
(\gamma_{\neg\exists{}M})

\\

\llbracket\lfloor\forall{}x.P(x)\rfloor{}\rrbracket & = &
\neg\lfloor\forall{}x.P(x)\rfloor\lor\lfloor{}P(t)\rfloor &
(\gamma_{\forall\mathrm{inst}})

\\

\llbracket\neg\lfloor\exists{}x.P(x)\rfloor{}\rrbracket & = &
\lfloor\exists{}x.P(x)\rfloor\lor\neg\lfloor{}P(t)\rfloor &
(\gamma_{\neg\exists\mathrm{inst}})
\end{array}$
\end{center}}
\caption{Rules of Tableau Theory}
\label{fig:tabth}
\end{figure}

\subsection{The Rewriting Theory}
\label{sec:rew}

A rewriting theory allows us to replace instantiations by
computations in the SMT solver. We aim to integrate rewriting in the broadest sense of the term as
proposed by deduction modulo theory. Deduction modulo theory~\cite{DA03} focuses
on the computational part of a theory, where axioms are transformed into rewrite
rules, which induces a congruence over formulas, and where reasoning is
performed modulo this congruence. In deduction modulo theory, this congruence is
then induced by a set of rewrite rules over both terms and formulas.

In the following, we borrow some of the notations and definitions
of~\cite{DA03}. We call $\mathrm{FV}$ the function that returns the set of
free variables of a term or a formula. A term rewrite rule is a pair of terms
denoted by $l\rew{}r$, where $\mathrm{FV}(r)\subseteq\mathrm{FV}(l)$. A
formula rewrite rule is a pair of formulas denoted by $l\rew{}r$, where
$l$ is an atomic formula and $r$ is an arbitrary formula, and where
$\mathrm{FV}(r)\subseteq\mathrm{FV}(l)$. A class rewrite system is a pair of
rewrite systems, denoted by $\mathcal{RE}$, consisting of $\mathcal{R}$, a set
of formula rewrite rules, and $\mathcal{E}$, a set of term rewrite rules.
Given a class rewrite system $\mathcal{RE}$, the relations $=_\mathcal{E}$ and
$=_\mathcal{RE}$ are the congruences generated respectively by the sets
$\mathcal{E}$ and $\mathcal{R}\cup\mathcal{E}$. In the following, we use the
standard concepts of subterm and term replacement: given an occurrence $\omega$
in a formula $P$, we write $P_{|\omega}$ for the term or formula at
$\omega$, and $P[t]_\omega$ for the formula obtained by replacing
$P_{|\omega}$ by $t$ in $P$ at $\omega$. Given a class rewrite system
$\mathcal{RE}$, the formula $P$ $\mathcal{RE}$-rewrites to $P'$, denoted by
$P\rew_\mathcal{RE}P'$, if $P=_\mathcal{E}Q$, $Q_{|\omega}=\sigma(l)$, and
$P'=_\mathcal{E}Q[\sigma(r)]_\omega$, for some rule $l\rew{}r\in\mathcal{R}$,
some formula $Q$, some occurrence $\omega$ in $Q$, and some substitution
$\sigma$. The relation $=_\mathcal{RE}$ is not decidable in general, but there are some
cases where this relation is decidable depending on the class rewrite system
$\mathcal{RE}$ and the rewrite relation $\rew_\mathcal{RE}$. In particular, if
the rewrite relation $\rew_\mathcal{RE}$ is confluent and (weakly) terminating,
then the relation $=_\mathcal{RE}$ is decidable.

The rewriting theory is integrated into the SMT solver in a similar way as for
the tableau theory. Whenever a literal is propagated or decided, we generate some clauses,
and add them. The clauses we generate
express the equivalence (resp. equality) between a formula (resp. term) and its normal form
in the rewrite system. More precisely, given a literal
$\lfloor{}P\rfloor$, where $P$ is a formula, and a formula $P'$ such that
$P=_\mathcal{RE}P'$, we generate and add the following clause:

$$\left(\bigvee_{(l,r)\in\mathcal{R}}
\neg\lfloor\forall{}\vec{x}.l\Leftrightarrow{}r\rfloor\right)\lor
\left(\bigvee_{(l,r)\in\mathcal{E}}\neg\lfloor\forall{}\vec{x}.l=r\rfloor\right)\lor
\lfloor{}P\Leftrightarrow{}P'\rfloor$$

where $\vec{x}=\mathrm{FV}(l)\cup\mathrm{FV}(r)$.

It should be noted that in usual SMT solvers, rewriting can be emulated by means
of triggers that are actually the left-hand side members of the class rewrite
system $\mathcal{RE}$ introduced above. But in our rewriting theory, we can
generate the formula resulting from the rewriting steps, while triggers can just
generate bindings, i.e. instances of the rewrite rules, which are used later to
relate the initial and rewritten formulas. Moreover, in our case, we can perform
several rewritings at once, while a trigger can only emulate one rewriting at a
time.

Let us illustrate the use of the rewriting theory by means of an example in set
theory. Let us prove that
$(\forall{}s,t.s\subseteq{}t\Leftrightarrow{}\forall{}x.x\in{}s\Rightarrow{}
x\in{}t) \Rightarrow{} {}a\subseteq{}a$, where $a$ is a constant. The proof is
given in Fig.~\ref{fig:rew}.
Note that, for the rewriting theory, any boxed quantified formula can be understood as
a rewrite rule as long as they represent one, for instance, the formula $B$ in
the example in Figure~\ref{fig:rew}.

\begin{figure}[t!]
\parbox{\textwidth}
{\small
\begin{center}
$\begin{array}{lcl}
\emptyset\parallel
\neg{}A & \longrightarrow & (\mathrm{unit~prop})\\

  \boldsymbol{\neg{}A}\parallel{}\mathcal{C}_1 = \neg{}A & \longrightarrow &
(\mathrm{Tableaux})\\

\neg{}A\parallel
\mathcal{C}_1,\boldsymbol{\mathcal{C}_2 = A\lor{}B},
\boldsymbol{\mathcal{C}_3 = A\lor\neg{}C}
& \longrightarrow & (\mathrm{unit~prop})\times{}2\\

\neg{}A,\boldsymbol{B},\boldsymbol{\neg{}C}\parallel
\mathcal{C}_1, \mathcal{C}_2, \mathcal{C}_3
& \longrightarrow & (\mathrm{Rewriting})\\

\neg{}A,B,\neg{}C\parallel
\mathcal{C}_1,\mathcal{C}_2,\mathcal{C}_3,
\boldsymbol{\mathcal{C}_4=\neg{}B\lor{}D}
& \longrightarrow & (\mathrm{unit~prop})\\

\neg{}A,B,\neg{}C,\boldsymbol{D}\parallel
\mathcal{C}_1,\mathcal{C}_2,\mathcal{C}_3,\mathcal{C}_4
& \longrightarrow & (\mathrm{Tableaux})\\

\neg{}A,B,\neg{}C,D\parallel
\mathcal{C}_1,\mathcal{C}_2,\mathcal{C}_3,\mathcal{C}_4,
\boldsymbol{\mathcal{C}_5=\neg{}D\lor{}E},
\boldsymbol{\mathcal{C}_6=\neg{}D\lor{}F}
& \longrightarrow & (\mathrm{unit~prop})\times{}2\\

\neg{}A,B,\neg{}C,D,\boldsymbol{E},\boldsymbol{F}\parallel
\mathcal{C}_1,\mathcal{C}_2,\mathcal{C}_3,\mathcal{C}_4, \mathcal{C}_5,
\mathcal{C}_6
& \longrightarrow & (\mathrm{Tableaux})\\

\neg{}A,B,\neg{}C,D,E,F\parallel
\mathcal{C}_1,\mathcal{C}_2,\mathcal{C}_3,\mathcal{C}_4,\mathcal{C}_5,
\mathcal{C}_6,\boldsymbol{\mathcal{C}_7=\neg{}F\lor\neg{}G\lor{}C}
& \longrightarrow & (\mathrm{unit~prop})\\

\neg{}A,B,\neg{}C,D,E,F,\boldsymbol{\neg{}G}\parallel
\mathcal{C}_1,\mathcal{C}_2,\mathcal{C}_3,\mathcal{C}_4, \mathcal{C}_5,
\mathcal{C}_6, \mathcal{C}_7
& \longrightarrow & (\mathrm{Tableaux})\\

\neg{}A,B,\neg{}C,D,E,F,\neg{}G\parallel
\mathcal{C}_1,\mathcal{C}_2,\mathcal{C}_3,\mathcal{C}_4, \mathcal{C}_5,
\mathcal{C}_6, \mathcal{C}_7,
  \boldsymbol{\mathcal{C}_8=G\lor\neg{}H}
& \longrightarrow & (\mathrm{unit~prop})\\

\neg{}A,B,\neg{}C,D,E,F,\neg{}G,\boldsymbol{\neg{}H}\parallel
\mathcal{C}_1,\mathcal{C}_2,\mathcal{C}_3,\mathcal{C}_4, \mathcal{C}_5,
\mathcal{C}_6, \mathcal{C}_7, \mathcal{C}_8
& \longrightarrow & (\mathrm{Tableaux})\\

\neg{}A,B,\neg{}C,D,E,F,\neg{}G,\neg{}H\parallel
\mathcal{C}_1,\mathcal{C}_2,\mathcal{C}_3,\mathcal{C}_4, \mathcal{C}_5,
\mathcal{C}_6, \mathcal{C}_7, \mathcal{C}_8, \\
~~~~\boldsymbol{\mathcal{C}_9=H\lor{}I},
\boldsymbol{\mathcal{C}_{10}=H\lor\neg{}I}
& \longrightarrow & (\mathrm{unit~prop})\\

\neg{}A,B,\neg{}C,D,E,F,\neg{}G,\neg{}H,\boldsymbol{I}\parallel
\mathcal{C}_1,\mathcal{C}_2,\mathcal{C}_3,\mathcal{C}_4, \mathcal{C}_5,
\mathcal{C}_6, \mathcal{C}_7, \mathcal{C}_8, \mathcal{C}_9, \mathcal{C}_{10}
& \longrightarrow & (\mathrm{unsat})\\

\boldsymbol{\mathrm{unsat}}
\end{array}$
\end{center}
\begin{flushleft}
$\begin{array}{l}
\mbox{where:}\\
\begin{array}{ll}
\multicolumn{2}{l}{
~~~~A\equiv\lfloor(\forall{}s,t.s\subseteq{}t\Leftrightarrow{}
\forall{}x.x\in{}s\Rightarrow{}x\in{}t)\Rightarrow{}a\subseteq{}a\rfloor}\\
~~~~B\equiv\lfloor\forall{}s,t.s\subseteq{}t\Leftrightarrow{}
\forall{}x.x\in{}s\Rightarrow{}x\in{}t\rfloor &
~~~~C\equiv\lfloor{}a\subseteq{}a\rfloor\\
~~~~D\equiv\lfloor{}a\subseteq{}a\Leftrightarrow
\forall{}x.x\in{}a\Rightarrow{}x\in{}a\rfloor &
~~~~E\equiv\lfloor{}a\subseteq{}a\Rightarrow
\forall{}x.x\in{}a\Rightarrow{}x\in{}a\rfloor\\
~~~~F\equiv\lfloor(\forall{}x.x\in{}a\Rightarrow{}x\in{}a)
\Rightarrow{}a\subseteq{}a\rfloor &
~~~~G\equiv\lfloor\forall{}x.x\in{}a\Rightarrow{}x\in{}a\rfloor\\
~~~~H\equiv\lfloor\epsilon_x\in{}a\Rightarrow\epsilon_x\in{}a\rfloor &
~~~~I\equiv\lfloor\epsilon_x\in{}a\rfloor
\end{array}\\\\
\mbox{with: }\epsilon_x=\epsilon(x).\neg(x\in{}a\Rightarrow{}x\in{}a)
\end{array}$
\end{flushleft}}
\caption{Example of Proof Using the Tableaux and Rewriting Theory}
\label{fig:rew}
\end{figure}

% $Id$

\section{Equational Reasoning with Rigid Unit Superposition}
\label{sec:super}

There are many ways of integrating equational reasoning in tableau
methods~\cite{DB75,LS02,BR15,DV96}. Because our prover does not rely on clausal
forms, but on arbitrary formulas with quantifiers occurring deep inside
branches, we deal with rigid variables, i.e. variables that can be instantiated
only once. The problem we want to solve, rigid E-unification modulo rewrite
rules, is the following. Assume a set of equations $E$, containing rigid
variables, a rewrite system $\mathcal{RE}$, and target terms $s$ and $t$. We
want a substitution $\sigma$ such that
$\bigwedge_{e \in E} e\sigma \vdash s\sigma =_\mathcal{E} t\sigma$. Such a
substitution is a solution to the rigid E-unification problem.

We propose here an approach based on superposition with rigid variables, as in
previous work by Degtyarev and Voronkov~\cite{DV96} and earlier work on rigid
paramodulation~\cite{DAP00}, but with significant differences. First, in order
to avoid constraint solving, we do not use basic superposition nor
constraints. Second, we introduce a merging rule, which factors together
intermediate (dis)equations that are alpha-equivalent: with multiple instances
of some of the quantified formulas (amplification), it becomes important not to
duplicate work. In this aspect, our calculus is quite close to labeled unit
superposition~\cite{KS10} when using sets as labels. Third, unlike rigid
paramodulation, we use a term ordering to orient the equations.

\subsection{Preliminary Definitions}

We write $ \clauseWithSubst{ s \approx t }{ \Sigma}$ (resp.
$\clauseWithSubst{ s \not\approx t }{ \Sigma}$), the unit clause that contains
exactly one equation (resp.~disequation) under hypothesis $\Sigma$ (which is a
set of substitutions). We write $\clauseWithSubst{\emptyset}{\Sigma}$ for the
empty clause under hypothesis $\Sigma$. We define $\renameVars{e}$, where $e$
is a (dis)equation, as follows: let $\sigma$ map every rigid variable of $e$ to
a fresh non-rigid variable, then
$\renameVars{e} = \clauseWithSubst{ e\sigma }{ \{ \sigma \} }$. For example,
$\renameVars{p(X)\approx a}$ is $\clauseWithSubst{ p(Y)\approx a}{ \{ X\mapsto
Y\} }$. The E-unification problem $E \vdash s\approx t$ can be solved by proving
$\clauseWithSubst{\emptyset }{ \Sigma}$ from
$\{ \renameVars{e} \}_{ e \in E } \cup \{ \renameVars{ s \not\approx t } \}$,
where $\Sigma$ contains the solutions. The meaning of $s \approx t | \Sigma$ is
that for every $\sigma \in \Sigma$, $s \approx t$ is provable using the
substitution $\sigma$ for the metavariables.

As can be noticed, we keep a set of substitutions, rather than unit clauses
paired with individual substitutions, in order to avoid duplicating the work for
alpha-equivalent clauses. Indeed, because of amplification, many instances of a
given (dis)equation might be present in a branch of the tableau. It would be
inefficient to repeat the same inference steps with each variant of the axioms.
Because we apply $\renameVars{e}$ on every initial $e$, clauses do not share any
variable, except in their attached sets of substitutions.

To perform an inference step between two unit (dis)equations, we merge their
sets of substitutions. Merging $\Sigma$ and $\Sigma'$, intuitively, means
computing $\{ \textsf{merge}(\sigma,\sigma') ~|~ \sigma \in \Sigma, \sigma'\in
\Sigma' \}$ for every pair $(\sigma,\sigma')$ of compatible substitutions. For
example, the resolution step between $p(x,x)| \{ X \mapsto a \}$ and
$\lnot p(y,b)| \{ X \mapsto y \}$ is not possible, because the result would need
to map $X$ to $a$ and $b$, which is impossible because $X$ is rigid.
Compatibility relies on a partial ordering $\leq$, such that
$\sigma \leq \sigma'$ means that $\sigma$ is less general than $\sigma'$.

Considering a substitution as a function from variables to terms, we define the
domain of a substitution $\sigma$ as the set of variables that have a
non-trivial binding in $\sigma$.\footnote{A trivial binding maps a variable to
itself.} The co-domain of a substitution is the set of variables occurring in
terms in the image of the domain of the substitution. In the following, we will
consider idempotent substitutions, i.e. substitutions for which the domain and
co-domain have an empty intersection.

The composition of two substitutions $\sigma$ and $\sigma'$, denoted by
$\sigma \circ \sigma'$, is said to be well-defined if and only if the domains of
$\sigma$ and $\sigma'$ have no intersection. In this case,
$\sigma \circ \sigma' \triangleq \left\{ x \mapsto (x\sigma)\sigma' | x \in
\text{domain}(\sigma) \right\}$. This definition extends to sets of
substitutions: $\Sigma \circ \sigma' \triangleq \left\{ \sigma \circ \sigma' |
\sigma \in \Sigma \right\}$. We then have $\sigma \leq \sigma'$ if and only if
$\exists \sigma''.~ \sigma \circ \sigma'' = \sigma'$. This notion also extends
to sets of substitutions: $\smash{ \Sigma \leq \Sigma' }$ if and only if
$\smash{ \forall \sigma' \in \Sigma'.~ \exists \sigma \in \Sigma. \sigma \leq
\sigma' }$. The merging of two substitutions $\sigma \uparrow \sigma'$ is the
supremum of $\{\sigma,\sigma'\}$ for the order $\leq$, if it exists, or $\bot$
otherwise. The merging of sets of substitutions is
$\Sigma \uparrow \Sigma' \triangleq \left\{ \sigma \uparrow \sigma' ~|~ \sigma
\in \Sigma, \sigma' \in \Sigma' \right., \sigma \uparrow \sigma' \not= \bot
\}$. An inference rule is said to be successful if the merging of the premises'
substitution sets is non-empty.

\subsection{Inference System}

In Fig.~\ref{fig:unit-sup-rules}, we present the rules for unit superposition
with rigid variables. We adopt notations and names from Schulz's paper on
E~\cite{SS02}. A single bar denotes an inference, i.e. we add the result to the
saturation set, whereas a double bar is a simplification in which the premises
are replaced by the conclusion(s). The relation $\prec$ is a reduction ordering,
used to orient equations and restrict inferences, thus pruning the search space.
Typically, $\prec$ is one of RPO or KBO. The rules of
Fig.~\ref{fig:unit-sup-rules} work as described below:

\begin{description}
\item[ER] is equality resolution, where a disequation
$\clauseWithSubst{s \not\approx t}\Sigma$ is solved by syntactically unifying
$s$ and $t$ with $\sigma$, if $\sigma$ is compatible with $\Sigma$.
\item[SN] is superposition into negative literals. A subterm of $u$ is rewritten
using $s \approx t$ after unifying it with $s$ by $\sigma$. The rewriting is
done only if $s\sigma \not\preceq t\sigma$, a sufficient (but not necessary)
condition for a ground instance of $s\sigma \approx t\sigma$ to be oriented
left-to-right.
\item[SP] is similar to SN, but superposes into a positive literal.
\item[TD1] deletes trivial equations that will never contribute to a proof.
\item[TD2] deletes clauses with an empty set of substitutions. In practice, we
only apply a rule if the conclusion is labeled with a non-empty set of
substitutions.
\item[ME] merges two alpha-equivalent clauses into a single clause, by merging
the sets of substitutions. This rule is very important in practice, to prevent
the search space from exploding due to the duplicates of most formulas.
Superposition deals with this explosion by removing duplicates using
subsumption, but in our context subsumption is not complete because rigid
variables are only proxy for ground terms: even if $C\sigma \subseteq D$, the
one ground instance of $C$ might not be compatible with the ground instance of
$D$.
\item[ES] is a restricted form of equality subsumption. The active equation
$\clauseWithSubst{ s\approx t}\Sigma $ can be used to delete another clause, as
in E~\cite{SS02}. However, ES only works if $s$ and $t$ are syntactically equal
to the corresponding subterms in the subsumed clause $C$. Otherwise, there is no
guarantee that further instantiations will not make $s\approx t$ incompatible
with $C$. Moreover, $C$ needs not be entirely removed. Only its substitutions
that are compatible with $\Sigma$ are subsumed.
\item[RP] is rewriting of positive clauses, which only works for syntactical
equality, not matching.
\item[RN] is the same as RP but for rewriting negative clauses.
\end{description}

Rule \textbf{SN} (resp.~\textbf{SP}) generates as many equations
(resp.~disequations) as there are in the set $(\Sigma \circ \sigma'')
\uparrow (\Sigma' \circ \sigma'')$ because all substitutions may not
always be merged, for instance consider $f(x, y) = t | \{
  \{ X_1 \mapsto g(x) \}, \{ X_2 \mapsto g(x) \}  \}$ and $f(z, a) = v | \{
  \{ X_1 \mapsto g(a) \} \} \}$, we have to derive two distinct, non-mergeable
equations
$ (t = v)\sigma_1 | \{ \sigma_1 = \{ X_1 \mapsto g(a) \} \}$ and
$ (t = v)\sigma_2 | \{ \sigma_2 = \{ X_1 \mapsto g(a); X_2 \mapsto g(x) \} \}$.

\begin{figure}[htb]
\begin{center}
% ER
\AXC{$s \not\approx t |\Sigma$}
\LL{ER}
\RL{if $\sigma = \text{mgu}(s, t)$}
\UIC{$\emptyset | \Sigma \circ \sigma $}\DP\\[12pt]

% SN
\AXC{$s \approx t | \Sigma$}
\AXC{$u \not\approx v | \Sigma'$}
\LL{SN}
\BIC{$\sigma''(u[p \leftarrow t] \not\approx v) | \sigma'''$}\DP
$\text{if} \left\{ \begin{array}{l@{\quad}l}
\sigma'' = \text{mgu}(u_{|p}, s) & u_{|p} \not\in V\\
\sigma''(s) \not\preceq \sigma''(t) & \sigma''(u) \not\preceq \sigma''(v)\\
\multicolumn{2}{l}{
\sigma''' \in (\Sigma \circ \sigma'') \uparrow (\Sigma' \circ \sigma'')}
\end{array}\right.$\\[12pt]

% SP
\AXC{$s \approx t | \Sigma$}
\AXC{$u \approx v | \Sigma'$}
\LL{SP}
\BIC{$\sigma''(u[p \leftarrow t] \approx v) | \sigma'''$}\DP
$\text{if} \left\{ \begin{array}{l@{\quad}l}
\sigma'' = \text{mgu}(u_{|p}, s) & u_{|p} \not\in V\\
\sigma''(s) \not\preceq \sigma''(t) & \sigma''(u) \not\preceq \sigma''(v)\\
\multicolumn{2}{l}{
\sigma''' \in (\Sigma \circ \sigma'') \uparrow (\Sigma' \circ \sigma'')}
\end{array}\right.$\\[12pt]

\mbox{
% TD1
\AXC{$s \approx s | \Sigma $}
\LL{TD1}
\doubleLine{}
\UIC{$\top$}\DP

% TD2
\AXC{$s \mathrel{R} t | \emptyset$}
\LL{TD2}
\RL{$ R \in \{ \approx, \not\approx \} $}
\doubleLine{}
\UIC{$\top$}
\DP}\\[12pt]

% ME
\AXC{$\rho(u) \approx \rho(v) | \Sigma$}
\AXC{$u \approx v | \Sigma'$}
\LL{ME}
\RL{$\rho \text{ is a variable renaming}$}
\doubleLine{}
\BIC{$\rho(u) \approx \rho(v) | \Sigma \cup (\Sigma' \circ \rho)$}\DP\\[12pt]

% ES
\AXC{$s \approx t | \Sigma$}
\AXC{$u[p \leftarrow s] \approx u[p \leftarrow t] | \Sigma' \cup \Sigma''$}
\LL{ES}
\RL{$
\text{if} \left\{ \begin{array}{l}
\Sigma'' \not= \emptyset\\
\Sigma \leq \Sigma''
\end{array}\right.$}
\doubleLine{}
\BIC{$s\approx t | \Sigma \qquad u[p\leftarrow s] \approx u[p \leftarrow t] |
\Sigma'$}\DP\\[12pt]

% RP
\AXC{$s \approx t | \Sigma$}
\AXC{$u \approx v | \Sigma'$}
\LL{RP}
\doubleLine{}
\BIC{$s \approx t | \Sigma$ \qquad $u[p \leftarrow t] \approx v | \Sigma'$}\DP
$\text{if} \left\{\begin{array}{l}
u_{|p} = s\\
s \succ t\\
\Sigma \leq \Sigma'\\
u \not\succeq v ~ \text{or} ~ p \neq \lambda\\
\end{array}\right.$\\[12pt]

% RN
\AXC{$s \approx t | \Sigma$}
\AXC{$u \not\approx v | \Sigma'$}
\LL{RN}
\doubleLine{}
\BIC{$s \approx t | \Sigma$ \qquad $u[p \leftarrow t] \not\approx v |
\Sigma'$}\DP
$\text{if} \left\{\begin{array}{l}
u_{|p} = s\\
s \succ t\\
\Sigma \leq \Sigma'\\
\end{array}\right.$
\caption{The Set of Rules for Unit Rigid Superposition}
\label{fig:unit-sup-rules}
\end{center}
\end{figure}

\subsection{Rewriting}

Rewrite rules can be integrated to the rigid unit superposition easily. In fact,
a rewrite rule $l\rew{}r$ can be expressed as an equality with a hypothesis set
consisting of a single trivial substitution
$s\approx{}t|\{\text{identity}\}$. Since the trivial substitution is compatible
with every substitution, it will never prevent any inference, thus allowing us
to use the unit clause as many times as needed to rewrite terms without
accumulating constraints, particularly using the rules RP and RN, whose side
conditions are always verified by rewrite rules. Rigid unit superposition
therefore provides an algorithm for rigid E-unification modulo rewrite rules.

\subsection{Main Loop}

Our objective with rigid E-unification is to attempt to close a branch of the
tableau prover (i.e. a set of Boolean literals set to true). To do so, all
equational or atomic literals are added to a set of unit clauses to process,
with a label $\Sigma \triangleq \{ \emptyset \}$. Then, the given-clause
algorithm is applied to try and saturate the set. Assuming a fair strategy, this
will eventually find a solution (i.e. derive
$\clauseWithSubst{\emptyset}{\Sigma}$) if there exists one. We refer the
interested reader to~\cite{SS02} for more details.

Because the whole branch is managed by a single given-clause saturation loop, we
look for all solutions susceptible to close the branch at the same time.
Moreover, this technique is amenable to incrementality, i.e. every time a
(dis)equation is decided by the SAT solver, we could add it to the saturation
set and perform a (limited) number of steps of the given-clause algorithm.

\subsection{Example}

To illustrate the calculus, we detail a refutation of the following set of
clauses stemming from set theory, where pair, fst, and snd are the constructor
and destructors of tuples, $f$ a function on tuples, and $X$ a rigid variable:

\[\begin{array}{rcl}
\text{pair}(\text{fst}(x), \text{snd}(x))) &\rew& x\\
\text{fst}(a) &\approx& \text{fst}(b)\\
p(a) &\not\approx& p(\text{pair}(\text{fst}(b), X))\\
\end{array}\]

Because the problem is purely equational, the tableau structure is trivial, and
all the work is done by the rigid superposition procedure as shown in
Fig.~\ref{fig:unit-sup-proof-example}.

\begin{figure}[t]
\begin{center}
\begin{tabular}{clc}
1 & axiom & $\text{pair}(\text{fst}(x), \text{snd}(x))) \rew x$\\

2 & axiom & $\text{fst}(a) = \text{fst}(b)$\\

3 & axiom & $p(a) \not= p(\text{pair}(\text{fst}(b), X))$\\

4 & \renameVarsSymb(1) &
$\clauseWithSubst
{ \text{pair}(\text{fst}(x), \text{snd}(x)) \approx x }
{ \{ \} }$\\

5 & \renameVarsSymb(2) &
$\clauseWithSubst
{ \text{fst}(a) \approx \text{fst}(b) }
{ \{ \} }$\\

6 & \renameVarsSymb(3) &
$\clauseWithSubst
{ f(a) \not\approx f(\text{pair}(\text{fst}(b), y)) }
{ \{ \mapVar{X}y \} }$\\

\midrule

7 & RN(5,6) &
$\clauseWithSubst
{ f(a) \not\approx f(\text{pair}(\text{fst}(a), y)) }
{ \{ \mapVar{X}y \} }$\\

8 & SN(4,7) &
$\clauseWithSubst
{ f(a) \not\approx f(a) }
{ \{ \mapVar{X}{\text{snd}(a)} \} }$\\

9 & ER(8) &
$\clauseWithSubst
{ \emptyset }
{ \{ \mapVar{X}{\text{snd}(a)} \} }$
\end{tabular}
\caption{Proof of a Set Theory Problem}
\label{fig:unit-sup-proof-example}
\end{center}
\end{figure}

% $Id$

\section{Implementation and Experimental Results}
\label{sec:bench}

In this section, we briefly describe the implementation of our approach
introduced previously, and present some experimental results obtained by running
this implementation over a benchmark of problems in the \bmth{} set theory.

\subsection{Implementation}

The algorithms described in this paper are implemented in the \archsat{}
automated theorem prover\footnote{Available at:
\url{https://gforge.inria.fr/projects/archsat}.}. It relies on the
\msat{}~\cite{GB17} library, derived from the \altergoz{} tool, and which is a
generic library for building automated deduction tools based on SAT solvers. The
SAT core in \msat{} is implemented using CDCL rather than the DPLL strategy
described in Sec.~\ref{sec:smt}, but our approach is actually agnostic with
respect to the SAT solver core and it therefore does not have any impact on the
first order reasoning or rewriting presented previously. \archsat{} (as well as
\msat{}) is written in \ocaml{}. \archsat{} natively supports polymorphic types
as described in~\cite{BP13}.

\subsection{Experimental Results}

As a framework to test our tool, we consider the set theory of the \bmth{}
method~\cite{B-Book}. This method is supported by some tool sets, such as
\atelierb{}, which are used in industry to specify and build, by stepwise
refinements, software that is correct by design. This theory is suitable as it
can be easily turned into a theory that is compatible with deduction modulo
theory, i.e. where a large part of axioms can be turned into rewrite rules, and
for which the rewriting theory proposed previously in Subsec.~\ref{sec:rew}
should work. Starting from the theory described in Chap.~2 of the
\bbook{}~\cite{B-Book}, we therefore transform whenever possible the axioms and
definitions into rewrite rules. The resulting theory has been introduced
in~\cite{BA15}, and due to lack of space, we only provide, in
Fig.~\ref{fig:bset}, the three rewriting rules corresponding to the axiomatic
core of the \bmth{} set theory that we consider.

As can be seen, the proposed theory is typed, using first order logic extended
to polymorphic types à la ML, through a type system in the spirit
of~\cite{BP13}. This extension to polymorphic types offers more flexibility, and
in particular allows us to deal with theories that rely on elaborate type
systems, like the \bmth{} set theory (see Chap.~2 of the
\bbook{}~\cite{B-Book}). The complete type system that is used in this
formalization can be found in~\cite{BA15}. The type constructors,
i.e. $\mathsf{tup}$ for tuples and $\mathsf{set}$ for sets, and type schemes of
the considered set constructs are provided in Fig.~\ref{fig:bset} as well, where
\type{} is the type of types and $\omicron$ the type of formulas, and where type
arguments are subscript annotations of the constructs.

\begin{figure}[t]
\small
\hspace{0.2cm}\underline{Axioms of Set Theory}
\begin{flushleft}
$\begin{array}{@{\hspace{0.2cm}}l}
(x,y)_{\alpha_1,\alpha_2}\in_{\tuple{\alpha_1}{\alpha_2}}s\times_{\alpha_1,\alpha_2}t\rew
x\in_{\alpha_1}s\land{}y\in_{\alpha_2}t\\
s\in_{\set{\alpha}}\mathbb{P}_\alpha(t)\rew
\forall{}x:\alpha.x\in_\alpha{}s\Rightarrow{}x\in_\alpha{}t\\
s=_{\set{\alpha}}t\rew
\forall{}x:\alpha.x\in_\alpha{}s\Leftrightarrow{}x\in_\alpha{}t
\end{array}$
\end{flushleft}
\hspace{0.2cm}\underline{Type Constructors}
\begin{flushleft}
$\begin{array}{@{\hspace{0.2cm}}l@{\hspace{1.0cm}}l}
\mathsf{tup}:\Pi\alpha_1,\alpha_2:\type.\type &
\mathsf{set}:\Pi\alpha:\type.\type
\end{array}$
\end{flushleft}
\hspace{0.2cm}\underline{Type Schemes of the Set Constructs}
\begin{flushleft}
$\begin{array}{@{\hspace{0.2cm}}lcl}
\arg\in\arg & : &
\Pi\alpha:\type.\alpha\rightarrow\set{\alpha}\rightarrow\omicron\\
(\arg,\arg) & : &
\Pi\alpha_1,\alpha_2:\type.\alpha_1\rightarrow\alpha_2\rightarrow
\tuple{\alpha_1}{\alpha_2}\\
\arg\times\arg & : &
\Pi\alpha_1,\alpha_2:\type.\set{\alpha_1}\rightarrow\set{\alpha_2}\rightarrow
\set{\tuple{\alpha_1}{\alpha_2}}\\
\mathbb{P}(\arg) & : &
\Pi\alpha:\type.\set{\alpha}\rightarrow\set{\set{\alpha}}\\
\arg=\arg & : & \Pi\alpha:\type.\alpha\rightarrow\alpha\rightarrow\omicron\\
\end{array}$
\end{flushleft}
\caption{Rewriting Rules of the Axiomatic Core of the \bmth{} Set Theory}
\label{fig:bset}
\end{figure}

To test \archsat{} in this theory, we consider 319~lemmas coming from Chap.~2 of
the \bbook{}~\cite{B-Book}. These lemmas are properties of various difficulty
regarding the set constructs introduced by the \bmth{} method. It should be
noted that these constructs and notations are, for a large part of them,
specific to the \bmth{} method, as they are used for the modeling of industrial
projects, and are not necessarily standard in set theory.

As tools, we consider \archsat{} (development version\footnote{\git{}
version~7720d8c.}). We also include other automated theorem provers, able to
deal with first order logic with polymorphic types and rewriting natively. In
particular, we consider \zipper{} (version~1.5), a prover based on superposition
and rewriting, as well as \zenm{} (version~0.4.2), a tableau-based prover that
is an extension of \zenon{} to deduction modulo theory. To show the impact of
rewriting over the results, we also include the \altergo{} SMT solver
(version~1.01). It would have been possible to also consider provers dealing
with pure first order logic and encode the polymorphic layer. But preliminary
tests have been conducted and very low results have been obtained even for the
best state-of-the-art provers (we have considered \e{} and \cvc{} in
particular), which tends to show that polymorphism encoding adds a lot of noise
in proof search and is not effective in practice.

The experiment was run on an \intel{}~3.50~GHz computer, with a timeout of 90~s
(beyond this timeout, results do not change) and a memory limit of 1~GiB. The
results are summarized in Tab.~\ref{tab:bench}. In these results, we observe
that \archsat{} obtains better results, in terms of proved problems, than
\zenm{} and \altergo{}, which tends to show the effectiveness of our approach in
practice. However, \archsat{} places second behind \zipper{}, which means that
there is still room for improvement regarding the implementation of our tool.
Looking at the cumulative times, \altergo{} is not really faster than
\archsat{} and \zipper{}, which take more time to find few more difficult
problems (with a timeout of 3~s, they respectively find 260 and 303~proofs in
16.61~s and 17.61~s, while \altergo{} obtains the same results).

\setlength{\tabcolsep}{3pt}
\renewcommand{\arraystretch}{1.2}
\newcolumntype{C}{>{\centering}X}

\begin{table}[t]
\begin{center}
\begin{tabularx}{\textwidth}{|X|C|C|C|C|}
\hline
\begin{tabular}{l}
319~Problems
\end{tabular} &
\begin{tabular}{c}
\archsat{}
\end{tabular} &
\begin{tabular}{c}
\zipper{}
\end{tabular} &
\begin{tabular}{c}
\zenm{}
\end{tabular} &
\begin{tabular}{c}
\altergo
\end{tabular}\tabularnewline
\hline
\begin{tabular}{l}
Proofs
\end{tabular} &
\begin{tabular}{c}
272
\end{tabular} &
\begin{tabular}{c}
306
\end{tabular} &
\begin{tabular}{c}
138
\end{tabular} &
\begin{tabular}{c}
232
\end{tabular}\tabularnewline
\hline
\begin{tabular}{l}
Rate
\end{tabular} &
\begin{tabular}{c}
85.3\%
\end{tabular} &
\begin{tabular}{c}
95,9\%
\end{tabular} &
\begin{tabular}{c}
43.3\%
\end{tabular} &
\begin{tabular}{c}
72.7\%
\end{tabular}\tabularnewline
\hline
\begin{tabular}{l}
Time \small{(s)}
\end{tabular} &
\begin{tabular}{c}
268.69
\end{tabular} &
\begin{tabular}{c}
109.88
\end{tabular} &
\begin{tabular}{c}
2.86
\end{tabular} &
\begin{tabular}{c}
8.42
\end{tabular}\tabularnewline
\hline
\end{tabularx}
\end{center}
\caption{Experimental Results over the \bmth{} Set Theory Benchmark}
\label{tab:bench}
\end{table}

\renewcommand{\arraystretch}{1}

% $Id$

\section{Conclusion}

We have described the architecture of \archsat{}, an automated theorem prover
that combines a SAT solver with tableau calculus and rewriting. Compared to
several other tools, \archsat{} appears quite effective in practice, as shown by
some experimental results obtained by running our implementation over a
benchmark of problems in the \bmth{} set theory.

As perspectives, we plan to realize more tests of \archsat{} over other theories
where a large part of these theories can be turned into rewrite rules. In
particular, a regular trigger mechanism has been also implemented in \archsat{}
and can be used to deal with conditional rewriting (the instantiation is delayed
and performed once the condition has been evaluated to true). This feature
should open up a range of new perspectives on the theories that our approach
could handle. We also aim to apply our tool to the benchmark of the \bware{}
project~\cite{BWare}, which consists of a large collection of proof obligations
coming from the development of industrial applications using the \bmth{} method.
This collection gathers about 13,000~problems, and should allow us to understand
to what extent our tool scales up, though it requires to extend \archsat{} to
handle arithmetic, which is why it hasn't been tested yet.


\bibliographystyle{abbrv}
\bibliography{biblio}

\end{document}
