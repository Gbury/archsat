% $Id$

\section{SAT Solving Modulo Tableau and Rewriting Theories}
\label{sec:smt}

Compared to genuine tableau automated theorem provers, like \princess{} or
\zenon{} for example, our approach has the benefit of being versatile since the
tableau rules are actually integrated as a regular SMT theory. This way, the
tableau rules can be easily combined with other theories, such as equality logic
with uninterpreted functions or arithmetic. The way we integrate the tableau
rules into the SMT solver (mainly by boxing/unboxing first order formulas) is
close to what is done in the \satallax{} tool~\cite{CEB12}. The difference
resides in the fact that we are in a pure first order framework, which has
significant consequences in the management of quantifiers and unification in
particular (see Sec.~\ref{sec:super}).

Regarding the integration of rewriting, automated theorem provers rely on
several solutions (superposition rule for first order provers, triggers for SMT
solvers, etc.). But deduction modulo theory~\cite{DA03} is probably the most
general approach, where a theory can be partly turned into a set of rewrite
rules over both terms and formulas. Several proof search methods have been
extended to deduction modulo theory, resulting in tools such as \iproverm{} and
\zenm{}. This paper can be seen as a continuation of these previous experiments
adapted to the framework of SMT solving.

\subsection{The Tableau Theory}
\label{sec:tab}

We introduce $\mathcal{T}$ and $\mathcal{F}$ respectively, the sets of first
order terms and formulas over the signature
$\mathcal{S}=(\mathcal{S}_\mathcal{F},\mathcal{S}_\mathcal{P})$, where
$\mathcal{S}_\mathcal{F}$ is the set of function symbols, and
$\mathcal{S}_\mathcal{P}$, the set of predicate symbols, such that
$\mathcal{S}_\mathcal{F}\cap\mathcal{S}_\mathcal{P}=\emptyset$. The set
$\mathcal{T}$ is extended with two kinds of terms specific to tableau proof
search. First are $\epsilon{}$-terms (used instead of Skolemization) of the form
$\epsilon(x).P(x)$, where $P(x)$ is a formula, and which means some $x$ that
satisfies $P(x)$, if it exists. And second, metavariables (often named free
variables in the tableau-related literature) of the form $X_P$, where $P$ is the
formula that introduces the metavariable, and which is either $\forall{}x.Q(x)$
or $\neg\exists{}x.Q(x)$, with $Q(x)$ a formula. Metavariables are written using
capitalized letters (such as $X_P$ or $Y_P$), will never be
substituted\footnote{More generally, terms are immutable and are never modified
in place: application of a substitution creates new terms that may be used,
but will never modify existing terms.}. They will also be considered rigid in
the following, meaning that in a context where we try and unify terms, by building
a mapping from metavariables to terms, each metavariables may be bound at most
once.  This is because in tableaux calculus, if you instantiate a formula
multiple times, it may create additional propositional branches, which we do not
want. Instead we consider metavariables rigid, and we'll generate multiple
(distinct) metavariables for the same formula if needed.

A boxed formula is of the form $\lfloor{}P\rfloor$, where $P$ is a formula. A
boxed formula is called an atom, and a literal is either an atom, or the
negation of an atom. A literal is such that there is no negation on top of the
boxed formula (which means that $\lfloor\neg{}P\rfloor$ is automatically
translated into $\neg\lfloor{}P\rfloor$, and $\neg\neg\lfloor{}P\rfloor$ into
$\lfloor{}P\rfloor$).  A clause is a disjunction of literals. It should be noted
that SMT solving usually reasons over sets of clauses composed of first order
terms; here, a literal is a first order formula (possibly with quantifiers),
which requires to box formulas to get a regular SAT solving problem where boxed
formulas are propositional variables.  The state of a SMT solver can be
represented as $T \parallel S$, with $T$ the trail of the SMT, i.e.~a list of
boxed formula, in chronological order left to right, and $S$ the set of clauses
to be satisfied by the solver. One very important point is that SMT theories can
and will often add some clauses to the set $S$ of clauses that the solver tries
to satisfy. This is sound as long as only tautologies are added. In the
following, this will often be abbreviated by only writing ``adding the clause''.

Tableau proof search method is integrated as a regular theory in our SMT solver.
Whenever a literal $l$ is decided or propagated by the solver, the tableaux
theory generates the set of clauses $\llbracket{}l\rrbracket$, where the
function $\llbracket\cdot\rrbracket$ is described by the rules of
Fig.~\ref{fig:tabth}, and then add these clauses\footnote{We use all rules
except the instantiation $\gamma$-rules: $\gamma_{\forall\mathrm{inst}}$ and
$\gamma_{\neg\exists\mathrm{inst}}$ which are only used once adequate terms
for instantiation have been found}.  It should be noted that we use the same
names for the rules as in tableau calculus ($\alpha$-rules, $\beta$-rules,
etc.), but there is no precedence between rules and therefore no priority in the
application of the rules contrary to the tableau proof search method (where
$\alpha$ rules are applied before $\beta$-rules, and so on).  Application of
this can be seen in Figure~\ref{fig:rew}, on the third line, where once the
solver has propagated that $A \equiv \lfloor B \rightarrow C \rfloor$ is false,
the tableaux theory adds two new clauses: $A \lor B$ and $A \lor \neg C$, whose
aim is to ensure that as long as $A$ is false, $B$ will be true and $C$ will be
false.  The clauses added should be seen as an implication: $\neg A \rightarrow
B$ and $\neg A \rightarrow \neg C$, where the presence of $A$ in the clauses is
important because the tableaux theory does not know whether $A$ is always false
or not: it might be that $A$ being false was a decision of the SAT solver (or a
consequence of a decision), and thus might be backtracked; in this case it is
important to not have the propagations of $B$ and $\neg C$ depend on the
propagation of $\neg A$.

When the SMT solver finds a model $M$ of the current set of clauses, we look for
a conflict in $M$ between boxed atomic formulas by unification.  If there exist
two literals $l$ and $\neg{}l'$ in $M$ such that $l=\lfloor{}Q\rfloor$ and
$l'=\lfloor{}R\rfloor$, with $Q$ and $R$ two formulas, then for each binding
$(X_{\forall{}x.P(x)}\mapsto{}t)\in\mathrm{mgu}(Q,R)$ (resp.
$(X_{\neg\exists{}x.P(x)}\mapsto{}t)\in\mathrm{mgu}(Q,R)$) we can generate the
clauses $\llbracket\lfloor\forall{}x.P(x)\rfloor{}\rrbracket$
(resp. $\llbracket\neg\lfloor\exists{}x.P(x)\rfloor{}\rrbracket$) using the rule
$\gamma_{\forall\mathrm{inst}}$ (resp. $\gamma_{\neg\exists\mathrm{inst}}$) of
Fig.~\ref{fig:tabth}, and then add these new clauses. For instance, suppose we
find a model where $\lfloor P(X_{\forall x. P(x)}) \rfloor$ is true and $\lfloor
P(a) \rfloor$ is false, then we'll add the following clause to the solver: $\neg
\lfloor \forall x. P(x) \rfloor \lor \lfloor P(a) \rfloor$, representing the
instantiation of $\forall x. P(x)$ using the term $a$. This clause will ensure
that the model where $\lfloor P(a) \rfloor$ is false cannot happen again as long
as $\lfloor \forall x. P(x) \rfloor$ is true. Note that we did not substitute
any metavariables, we only added some tautologies, thus there is no need for any
special backtracking.

\begin{figure}[t]
\parbox{\textwidth}
{\small
\underline{Analytic Rules}
\begin{center}
$\begin{array}{llll}
(\alpha_\land) &
\llbracket\lfloor{}P\land{}Q\rfloor{}\rrbracket =
\left\{\begin{array}{l}
  \neg\lfloor{}P\land{}Q\rfloor\lor\lfloor{}P\rfloor \\
  \neg\lfloor{}P\land{}Q\rfloor\lor\lfloor{}Q\rfloor
\end{array}\right. &

(\beta_{\neg\land}) &
\llbracket\neg\lfloor{}P\land{}Q\rfloor{}\rrbracket =
\lfloor{}P\land{}Q\rfloor\lor\neg\lfloor{}P\rfloor\lor\neg\lfloor{}Q\rfloor

\\\\

(\beta_\lor) &
\llbracket{}\lfloor{}P\lor{}Q\rfloor{}\rrbracket =
\neg\lfloor{}P\lor{}Q\rfloor\lor\lfloor{}P\rfloor\lor\lfloor{}Q\rfloor &

(\alpha_{\neg\lor}) &
\llbracket\neg\lfloor{}P\lor{}Q\rfloor{}\rrbracket =
\left\{\begin{array}{l}
  \lfloor{}P\lor{}Q\rfloor\lor\neg\lfloor{}P\rfloor \\
  \lfloor{}P\lor{}Q\rfloor\lor\neg\lfloor{}Q\rfloor
\end{array}\right.

\\\\

(\beta_\Rightarrow) &
\llbracket{}\lfloor{}P\Rightarrow{}Q\rfloor{}\rrbracket =
\neg\lfloor{}P\Rightarrow{}Q\rfloor\lor\neg\lfloor{}P\rfloor\lor
\lfloor{}Q\rfloor &

(\alpha_{\neg\Rightarrow}) &
\llbracket\neg\lfloor{}P\Rightarrow{}Q\rfloor{}\rrbracket =
\left\{\begin{array}{l}
  \lfloor{}P\Rightarrow{}Q\rfloor\lor\lfloor{}P\rfloor \\
  \lfloor{}P\Rightarrow{}Q\rfloor\lor\neg\lfloor{}Q\rfloor
\end{array}\right.

\\\\

(\beta_\Rightarrow) &
\llbracket{}\lfloor{}P\Leftrightarrow{}Q\rfloor{}\rrbracket =
\left\{\begin{array}{l}
  \neg\lfloor{}P\Leftrightarrow{}Q\rfloor\lor\lfloor{}P\Rightarrow{}Q\rfloor \\
  \neg\lfloor{}P\Leftrightarrow{}Q\rfloor\lor\lfloor{}Q\Rightarrow{}P\rfloor
\end{array}\right. &

(\beta_{\neg\Rightarrow}) &
\llbracket{}\neg\lfloor{}P\Leftrightarrow{}Q\rfloor{}\rrbracket =
\begin{array}{l}
  \lfloor{}P\Leftrightarrow{}Q\rfloor \\
  \lor\neg\lfloor{}P\Rightarrow{}Q\rfloor \\
  \lor\neg\lfloor{}Q\Rightarrow{}P\rfloor
\end{array}
\end{array}$
\end{center}

\underline{$\delta$-Rules}
\begin{center}
$\begin{array}{lll@{\hspace{0.5cm}}l}
\llbracket\lfloor\exists{}x.P(x)\rfloor{}\rrbracket & = &
\neg{}\lfloor\exists{}x.P(x)\rfloor\lor\lfloor{}P(\epsilon(x).P(x))\rfloor &
(\delta_\exists)

\\

\llbracket\neg\lfloor\forall{}x.P(x)\rfloor{}\rrbracket & = &
\lfloor\forall{}x.P(x)\rfloor\lor\neg\lfloor{}P(\epsilon(x).\neg{}P(x))\rfloor &
(\delta_{\neg\forall})
\end{array}$
\end{center}

\underline{$\gamma$-Rules}
\begin{center}
$\begin{array}{lll@{\hspace{0.5cm}}l}
\llbracket\lfloor\forall{}x.P(x)\rfloor{}\rrbracket & = &
\neg\lfloor\forall{}x.P(x)\rfloor\lor\lfloor{}P(X_{\forall{}x.P(x)})\rfloor &
(\gamma_{\forall{}M})

\\

\llbracket\neg\lfloor\exists{}x.P(x)\rfloor{}\rrbracket & = &
\lfloor\exists{}x.P(x)\rfloor\lor
\neg\lfloor{}P(X_{\neg\exists{}x.P(x)})\rfloor &
(\gamma_{\neg\exists{}M})

\\

\llbracket\lfloor\forall{}x.P(x)\rfloor{}\rrbracket & = &
\neg\lfloor\forall{}x.P(x)\rfloor\lor\lfloor{}P(t)\rfloor &
(\gamma_{\forall\mathrm{inst}})

\\

\llbracket\neg\lfloor\exists{}x.P(x)\rfloor{}\rrbracket & = &
\lfloor\exists{}x.P(x)\rfloor\lor\neg\lfloor{}P(t)\rfloor &
(\gamma_{\neg\exists\mathrm{inst}})
\end{array}$
\end{center}}
\caption{Rules of Tableau Theory}
\label{fig:tabth}
\end{figure}

\subsection{The Rewriting Theory}
\label{sec:rew}

A rewriting theory allows us to replace instantiations by
computations in the SMT solver. We aim to integrate rewriting in the broadest sense of the term as
proposed by deduction modulo theory. Deduction modulo theory~\cite{DA03} focuses
on the computational part of a theory, where axioms are transformed into rewrite
rules, which induces a congruence over formulas, and where reasoning is
performed modulo this congruence. In deduction modulo theory, this congruence is
then induced by a set of rewrite rules over both terms and formulas.

In the following, we borrow some of the notations and definitions
of~\cite{DA03}. We call $\mathrm{FV}$ the function that returns the set of
free variables of a term or a formula. A term rewrite rule is a pair of terms
denoted by $l\rew{}r$, where $\mathrm{FV}(r)\subseteq\mathrm{FV}(l)$. A
formula rewrite rule is a pair of formulas denoted by $l\rew{}r$, where
$l$ is an atomic formula and $r$ is an arbitrary formula, and where
$\mathrm{FV}(r)\subseteq\mathrm{FV}(l)$. A class rewrite system is a pair of
rewrite systems, denoted by $\mathcal{RE}$, consisting of $\mathcal{R}$, a set
of formula rewrite rules, and $\mathcal{E}$, a set of term rewrite rules.
Given a class rewrite system $\mathcal{RE}$, the relations $=_\mathcal{E}$ and
$=_\mathcal{RE}$ are the congruences generated respectively by the sets
$\mathcal{E}$ and $\mathcal{R}\cup\mathcal{E}$. In the following, we use the
standard concepts of subterm and term replacement: given an occurrence $\omega$
in a formula $P$, we write $P_{|\omega}$ for the term or formula at
$\omega$, and $P[t]_\omega$ for the formula obtained by replacing
$P_{|\omega}$ by $t$ in $P$ at $\omega$. Given a class rewrite system
$\mathcal{RE}$, the formula $P$ $\mathcal{RE}$-rewrites to $P'$, denoted by
$P\rew_\mathcal{RE}P'$, if $P=_\mathcal{E}Q$, $Q_{|\omega}=\sigma(l)$, and
$P'=_\mathcal{E}Q[\sigma(r)]_\omega$, for some rule $l\rew{}r\in\mathcal{R}$,
some formula $Q$, some occurrence $\omega$ in $Q$, and some substitution
$\sigma$. The relation $=_\mathcal{RE}$ is not decidable in general, but there are some
cases where this relation is decidable depending on the class rewrite system
$\mathcal{RE}$ and the rewrite relation $\rew_\mathcal{RE}$. In particular, if
the rewrite relation $\rew_\mathcal{RE}$ is confluent and (weakly) terminating,
then the relation $=_\mathcal{RE}$ is decidable.

The rewriting theory is integrated into the SMT solver in a similar way as for
the tableau theory. Whenever a literal is propagated or decided, we generate some clauses,
and add them. The clauses we generate
express the equivalence (resp. equality) between a formula (resp. term) and its normal form
in the rewrite system. More precisely, given a literal
$\lfloor{}P\rfloor$, where $P$ is a formula, and a formula $P'$ such that
$P=_\mathcal{RE}P'$, we generate and add the following clause:

$$\left(\bigvee_{(l,r)\in\mathcal{R}}
\neg\lfloor\forall{}\vec{x}.l\Leftrightarrow{}r\rfloor\right)\lor
\left(\bigvee_{(l,r)\in\mathcal{E}}\neg\lfloor\forall{}\vec{x}.l=r\rfloor\right)\lor
\lfloor{}P\Leftrightarrow{}P'\rfloor$$

where $\vec{x}=\mathrm{FV}(l)\cup\mathrm{FV}(r)$.

It should be noted that in usual SMT solvers, rewriting can be emulated by means
of triggers that are actually the left-hand side members of the class rewrite
system $\mathcal{RE}$ introduced above. But in our rewriting theory, we can
generate the formula resulting from the rewriting steps, while triggers can just
generate bindings, i.e. instances of the rewrite rules, which are used later to
relate the initial and rewritten formulas. Moreover, in our case, we can perform
several rewritings at once, while a trigger can only emulate one rewriting at a
time.

Let us illustrate the use of the rewriting theory by means of an example in set
theory. Let us prove that
$(\forall{}s,t.s\subseteq{}t\Leftrightarrow{}\forall{}x.x\in{}s\Rightarrow{}
x\in{}t) \Rightarrow{} {}a\subseteq{}a$, where $a$ is a constant. The proof is
given in Fig.~\ref{fig:rew}.
Note that, for the rewriting theory, any boxed quantified formula can be understood as
a rewrite rule as long as they represent one, for instance, the formula $B$ in
the example in Figure~\ref{fig:rew}.

\begin{figure}[t!]
\parbox{\textwidth}
{\small
\begin{center}
$\begin{array}{lcl}
\emptyset\parallel
\neg{}A & \longrightarrow & (\mathrm{unit~prop})\\

  \boldsymbol{\neg{}A}\parallel{}\mathcal{C}_1 = \neg{}A & \longrightarrow &
(\mathrm{Tableaux})\\

\neg{}A\parallel
\mathcal{C}_1,\boldsymbol{\mathcal{C}_2 = A\lor{}B},
\boldsymbol{\mathcal{C}_3 = A\lor\neg{}C}
& \longrightarrow & (\mathrm{unit~prop})\times{}2\\

\neg{}A,\boldsymbol{B},\boldsymbol{\neg{}C}\parallel
\mathcal{C}_1, \mathcal{C}_2, \mathcal{C}_3
& \longrightarrow & (\mathrm{Rewriting})\\

\neg{}A,B,\neg{}C\parallel
\mathcal{C}_1,\mathcal{C}_2,\mathcal{C}_3,
\boldsymbol{\mathcal{C}_4=\neg{}B\lor{}D}
& \longrightarrow & (\mathrm{unit~prop})\\

\neg{}A,B,\neg{}C,\boldsymbol{D}\parallel
\mathcal{C}_1,\mathcal{C}_2,\mathcal{C}_3,\mathcal{C}_4
& \longrightarrow & (\mathrm{Tableaux})\\

\neg{}A,B,\neg{}C,D\parallel
\mathcal{C}_1,\mathcal{C}_2,\mathcal{C}_3,\mathcal{C}_4,
\boldsymbol{\mathcal{C}_5=\neg{}D\lor{}E},
\boldsymbol{\mathcal{C}_6=\neg{}D\lor{}F}
& \longrightarrow & (\mathrm{unit~prop})\times{}2\\

\neg{}A,B,\neg{}C,D,\boldsymbol{E},\boldsymbol{F}\parallel
\mathcal{C}_1,\mathcal{C}_2,\mathcal{C}_3,\mathcal{C}_4, \mathcal{C}_5,
\mathcal{C}_6
& \longrightarrow & (\mathrm{Tableaux})\\

\neg{}A,B,\neg{}C,D,E,F\parallel
\mathcal{C}_1,\mathcal{C}_2,\mathcal{C}_3,\mathcal{C}_4,\mathcal{C}_5,
\mathcal{C}_6,\boldsymbol{\mathcal{C}_7=\neg{}F\lor\neg{}G\lor{}C}
& \longrightarrow & (\mathrm{unit~prop})\\

\neg{}A,B,\neg{}C,D,E,F,\boldsymbol{\neg{}G}\parallel
\mathcal{C}_1,\mathcal{C}_2,\mathcal{C}_3,\mathcal{C}_4, \mathcal{C}_5,
\mathcal{C}_6, \mathcal{C}_7
& \longrightarrow & (\mathrm{Tableaux})\\

\neg{}A,B,\neg{}C,D,E,F,\neg{}G\parallel
\mathcal{C}_1,\mathcal{C}_2,\mathcal{C}_3,\mathcal{C}_4, \mathcal{C}_5,
\mathcal{C}_6, \mathcal{C}_7,
  \boldsymbol{\mathcal{C}_8=G\lor\neg{}H}
& \longrightarrow & (\mathrm{unit~prop})\\

\neg{}A,B,\neg{}C,D,E,F,\neg{}G,\boldsymbol{\neg{}H}\parallel
\mathcal{C}_1,\mathcal{C}_2,\mathcal{C}_3,\mathcal{C}_4, \mathcal{C}_5,
\mathcal{C}_6, \mathcal{C}_7, \mathcal{C}_8
& \longrightarrow & (\mathrm{Tableaux})\\

\neg{}A,B,\neg{}C,D,E,F,\neg{}G,\neg{}H\parallel
\mathcal{C}_1,\mathcal{C}_2,\mathcal{C}_3,\mathcal{C}_4, \mathcal{C}_5,
\mathcal{C}_6, \mathcal{C}_7, \mathcal{C}_8, \\
~~~~\boldsymbol{\mathcal{C}_9=H\lor{}I},
\boldsymbol{\mathcal{C}_{10}=H\lor\neg{}I}
& \longrightarrow & (\mathrm{unit~prop})\\

\neg{}A,B,\neg{}C,D,E,F,\neg{}G,\neg{}H,\boldsymbol{I}\parallel
\mathcal{C}_1,\mathcal{C}_2,\mathcal{C}_3,\mathcal{C}_4, \mathcal{C}_5,
\mathcal{C}_6, \mathcal{C}_7, \mathcal{C}_8, \mathcal{C}_9, \mathcal{C}_{10}
& \longrightarrow & (\mathrm{unsat})\\

\boldsymbol{\mathrm{unsat}}
\end{array}$
\end{center}
\begin{flushleft}
$\begin{array}{l}
\mbox{where:}\\
\begin{array}{ll}
\multicolumn{2}{l}{
~~~~A\equiv\lfloor(\forall{}s,t.s\subseteq{}t\Leftrightarrow{}
\forall{}x.x\in{}s\Rightarrow{}x\in{}t)\Rightarrow{}a\subseteq{}a\rfloor}\\
~~~~B\equiv\lfloor\forall{}s,t.s\subseteq{}t\Leftrightarrow{}
\forall{}x.x\in{}s\Rightarrow{}x\in{}t\rfloor &
~~~~C\equiv\lfloor{}a\subseteq{}a\rfloor\\
~~~~D\equiv\lfloor{}a\subseteq{}a\Leftrightarrow
\forall{}x.x\in{}a\Rightarrow{}x\in{}a\rfloor &
~~~~E\equiv\lfloor{}a\subseteq{}a\Rightarrow
\forall{}x.x\in{}a\Rightarrow{}x\in{}a\rfloor\\
~~~~F\equiv\lfloor(\forall{}x.x\in{}a\Rightarrow{}x\in{}a)
\Rightarrow{}a\subseteq{}a\rfloor &
~~~~G\equiv\lfloor\forall{}x.x\in{}a\Rightarrow{}x\in{}a\rfloor\\
~~~~H\equiv\lfloor\epsilon_x\in{}a\Rightarrow\epsilon_x\in{}a\rfloor &
~~~~I\equiv\lfloor\epsilon_x\in{}a\rfloor
\end{array}\\\\
\mbox{with: }\epsilon_x=\epsilon(x).\neg(x\in{}a\Rightarrow{}x\in{}a)
\end{array}$
\end{flushleft}}
\caption{Example of Proof Using the Tableaux and Rewriting Theory}
\label{fig:rew}
\end{figure}
